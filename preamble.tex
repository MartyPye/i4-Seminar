\documentclass{acm_proc_article-sp}


% Load basic packages
\usepackage{balance}  % to better equalize the last page
\usepackage{graphics} % for EPS, load graphicx instead
% \usepackage{times}    % comment if you want LaTeX's default font
\usepackage{url}      % llt: nicely formatted URLs
\usepackage{color}    % colors for comments
\usepackage{latexsym} % for checkboxes
\usepackage{tipa}  % for chi letter
\usepackage{subfigure} %subfigure
\usepackage{marvosym}   % Euro symbol
\usepackage{amsmath}  % math symbols
% \usepackage{caption}
\usepackage{subfigure}
\usepackage{url}

% llt: Define a global style for URLs, rather that the default one
\makeatletter
\def\url@leostyle{%
  \@ifundefined{selectfont}{\def\UrlFont{\sf}}{\def\UrlFont{\small\bf\ttfamily}}}
\makeatother
\urlstyle{leo}

% get rid of the copyright box.
\makeatletter
\let\@copyrightspace\relax
\makeatother


% To make various LaTeX processors do the right thing with page size.
\def\pprw{8.5in}
\def\pprh{11in}
\special{papersize=\pprw,\pprh}
\setlength{\paperwidth}{\pprw}
\setlength{\paperheight}{\pprh}
\setlength{\pdfpagewidth}{\pprw}
\setlength{\pdfpageheight}{\pprh}

% Make sure hyperref comes last of your loaded packages, 
% to give it a fighting chance of not being over-written, 
% since its job is to redefine many LaTeX commands.
\usepackage{hyperref}
% \usepackage[pdftex]{hyperref}
% \hypersetup{
% pdftitle={Interacting with Smart(er) Objects},
% pdfauthor={Marty Pye},
% bookmarksnumbered,
% pdfstartview={FitH},
% colorlinks,
% citecolor=black,
% filecolor=black,
% linkcolor=black,
% urlcolor=black,
% breaklinks=true,
% }

% create a shortcut to typeset table headings
\newcommand\tabhead[1]{\small\textbf{#1}}

% editing commands
\newcommand{\comment}[1]{}
\definecolor{Orange}{rgb}{1,0.5,0}
\newcommand{\todo}[1]{\textsf{\textbf{\textcolor{Orange}{[[#1]]}}}}
\newcommand{\bug}[1]{\textsf{\textcolor{Orange}{#1}}}


% check boxes
\newcommand*{\thecheckbox}{\hss$\Box$} 
\newenvironment*{checklist} 
{\list{}{% 
\renewcommand*{\makelabel}[1]{\thecheckbox}}} 
{\endlist} 

% inch symbol
\def\inch#1{#1$''$}

% subscript
\newcommand{\superscript}[1]{\ensuremath{^{\textrm{#1}}}}
\newcommand{\subscript}[1]{\ensuremath{_{\textrm{#1}}}}

% Images
% \myFigure	[ LABEL_PREFIX (optional) ]
%			{ FILENAME (without extension) }
%			{ CAPTION TEXT }
%			{ SHORT VERSION OF CAPTION TEXT }
\newcommand{\myFigure}[2]
{%
	\begin{figure}[!h]
		\centering
		\includegraphics[width= 0.95\columnwidth]{#1}
		\caption{#2}
		\label{#1}
	\end{figure}
}

% Footnote without symbol
\newcommand\blfootnote[1]{%
  \begingroup
  \renewcommand\thefootnote{}\footnote{#1}%
  \addtocounter{footnote}{-1}%
  \endgroup
}



% statistics report
\def\ANOVA#1#2#3#4{$F_{#1,#2} = #3, p = #4$}
\def\ANOVAS#1#2#3#4{$F_{#1,#2} = #3, p < #4$}
\def\tt#1#2#3{$t_{#1} = #2, p = #3$}
\def\tts#1#2#3{$t_{#1} = #2, p < #3$}
\def\Wilcox#1#2{$Z = #1, p #2$}
\def\GEE#1#2#3#4{Wald \textchi$^{2}_{#1,N=#2}=#3,\ p#4$}
\def\Mean#1{($M = #1$)}


% variable names
\def\PointingTime{\emph{ManipulationTime}}
\def\DriftCount{\emph{DriftCount}}
\def\Thickness{\emph{Thickness}}
\def\MovementDistance{\emph{MovementDistance}}
\def\SurfaceSupport{\emph{SurfaceSupport}}
\def\UserID{\emph{UserID}}
\def\SmallMovement{\emph{SmallMovement}}
\def\LargeMovement{\emph{LargeMovement}}
\def\HandOnSurface{\emph{HandOnSurface}}
\def\HandInMidair{\emph{HandInMidair}}
\def\Predepressing{\emph{Pre-depressing}}
\def\DragDepressing{\emph{DragDepressing}}
\def\ClickDepressing{\emph{ClickDepressing}}
\def\PreLeaving{\emph{Pre-leaving}}
\def\Leaving{\emph{Leaving}}

% End of preamble. Here it comes the document.
\begin{document}

\title{Interacting with Smart(er) objects}

\numberofauthors{1}
\author{Marty Pye\\ RWTH Aachen University\\ marty.pye@rwth-aachen.de}

\maketitle
\begin{abstract}
The Internet of Things (IoT) is currently a rapidly growing field in computer science.
The amount of internet enabled devices and objects is growing vastly. 
The IoT concept is much broader than the familiar collection of internet enabled devices such as laptops, smartphones, tablets, etc.
It includes all sorts of everyday objects that previously did not contain electronic components, but are now sensing and communicating with each other and the internet.

Careful consideration needs to be put into the interaction design concept of such objects, such that it does not interfere with the original purpose of the object, or at least merges well with that purpose.
Several challenges need to be met when embedding interaction in everyday objects.
These challenges include the invisibility dilemma, embedded vs.\ dedicated device interaction and implicit vs.\ explicit interaction, and are discussed in Section~\ref{sec:interaction}.
How these challenges have been tackled by previous researchers in their systems is then elaborated in Section~\ref{sec:caseStudies}. 
This thesis then concludes with a derived set of guidelines that developers in this area should take into account, so as not to succumb to the common pitfalls of designing interactive everyday objects.

The other component which needs consideration is the actual implementation of the connectivity and interactivity. Depending on the environment and context of the smart object, different protocols make more sense.
This thesis briefly presents the different technologies in Section~\ref{sec:techniques}.
On the basis of several research projects, the thesis then discusses, which communication protocols would make more sense if those projects were implemented in real-world scenarios.
\end{abstract}